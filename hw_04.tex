\documentclass[%
  10pt,
  letterpaper
]{article}
\usepackage{amsmath}
\usepackage{amssymb}
\usepackage{amsfonts}
\usepackage{IEEEtrantools}
\usepackage{MnSymbol}
\DeclareMathOperator\lcm{lcm}
% \usepackage{tikz}
\IEEEeqnarraydefcolsep{0}{\leftmargini}
\usepackage{bbm}
\setlength\parindent{0pt}
\setlength\parskip{4pt}
\begin{document}
\begin{center}
{UNIVERSITY OF SOUTH CAROLINA}
\end{center}
\large{MATH-546 Algebraic Structures I\hfill Homework 4}
\subsection*{Problem 1: }
Let $G$ be a group and $H, K$ subgroups of $G$.
\begin{itemize}
\item[a.] Prove that $H \cap K$ is also a subgroup of $G$.
\item[b.] Assume that $H \cup K$ is a subgroup of $G$. Prove that $H \subseteq K$ or $K \subseteq H$. 
\end{itemize}

\subsection*{Answer 1:}
\begin{itemize}
\item[a.]
  $\forall a,b \in H\cap K$:
  \begin{IEEEeqnarray*}{0rCll}
    b\in H\cap K&  \Rightarrow & (b^{-1}\in H) \land (b^{-1}\in K)&\quad\textrm{closure}\\
     & \Rightarrow & b^{-1} \in (H\cap K)\\
    & \Rightarrow & (ab^{-1} \in H) \land  (ab^{-1} \in K) &\quad\textrm{closure}\\
     & \Rightarrow & ab^{-1} \in (H \cap K)\\
    & \Rightarrow & H\cap K \leq G&\quad\textrm{TP 1}
  \end{IEEEeqnarray*}
\item[b.] Given $H\leq G, K\leq G, (H\cup K)\leq G$:
  \[\neg(( H \subseteq K) \lor( K \subseteq H)) \Rightarrow \exists h\in (H \setminus K)\land
  \exists k\in (K\setminus H)\]
  So consider $hk$:
  \begin{IEEEeqnarray*}{0rCll}
    hk \in H\cup K & & &\quad\textrm{closure of}\ H\cup K\\
    hk \in H\cup K & \Rightarrow & (hk \in H) \lor( hk \in K)\\
    hk \in H & \Rightarrow & h^{-1}hk \in H &\quad\textrm{closure of}\ H\\
     & \Rightarrow & k \in H \\
    hk \in K & \Rightarrow & hkk^{-1} \in K &\quad\textrm{closure of}\ K\\
     & \Rightarrow & h \in K \\
  \end{IEEEeqnarray*}

\end{itemize}
\newpage
\subsection*{Problem 2: }
Let $G$ be a group and $a \in G$ a fixed element. Let
    \[
        H = \{ x \in G \mid ax = xa \}.
    \]
Prove that $H$ is a subgroup of $G$.
    
\subsection*{Answer 2:}
\begin{itemize}
\item[] Associativity: inherited
\item[] Identity: $ae = ea \Rightarrow e \in H$
\item[] Inverses: WTS: $x\in H \Rightarrow x^{-1} \in H$:
  \begin{IEEEeqnarray*}{0rCl}
    xx^{-1} & = & x^{-1}x\\
    axx^{-1} & = & ax^{-1}x\\
    xax^{-1} & = & ae \\
    x^{-1}xax^{-1} & = & x^{-1}a\\
    ax^{-1} & = & x^{-1}a
  \end{IEEEeqnarray*}
\item[] Closure: $\forall x,y \in H: a(xy) = xay = (xy)a$
\end{itemize}
\newpage
\subsection*{Problem 3: }
Let $G$ be a group. The center of $G$ is defined as
\[
Z(G) = \{ x \in G \mid ax = xa \;\; \forall a \in G \}.
\]
\begin{itemize}
  \item[a.] Prove that $Z(G)$ is a subgroup of $G$.
  \item[b.] Let $G = GL_2(\mathbb{R})$ (the group of $2 \times 2$ invertible matrices with
    matrix multiplication as operation). Prove that
    \[
    Z(G) = \left\{ 
    \begin{bmatrix}
      c & 0 \\
      0 & c
    \end{bmatrix}
    \;\middle|\; c \neq 0 \right\}.
    \]
\end{itemize}
\subsection*{Answer 3:}
\begin{itemize}
\item[a.] Let $H_a = \{x \in G \mid ax=xa\}$ then  $H_a$ is a subgroup of $G$ (by Problem 2).
Then by problem 1a:
  \[ Z(G) = \{ x \in G \mid ax = xa \;\; \forall a \in G \} = \bigcap_{a\in G}H_a\]
  is a group.
\item[b.]  Clearly $\begin{bmatrix}
      c & 0 \\
      0 & c
    \end{bmatrix} = c \begin{bmatrix}
      1 & 0 \\
      0 & 1
    \end{bmatrix} \in Z(G)$ since scalars and the identity commute with matrices in $GL_2(\mathbb{R})$.

  Let $\begin{bmatrix}
      a & b \\
      c & d
    \end{bmatrix} \in  Z(G)$ and using invertible matrix $\begin{bmatrix}
      1 & 0 \\
      0 & -1
\end{bmatrix}$:
\begin{IEEEeqnarray*}{0rCl}\begin{bmatrix}
      a & b \\
      c & d
    \end{bmatrix}\begin{bmatrix}
      1 & 0 \\
      0 & -1
  \end{bmatrix} & = & \begin{bmatrix}
  a & -b\\c & -d
  \end{bmatrix}\\
  \begin{bmatrix}
      1 & 0 \\
      0 & -1
  \end{bmatrix} \begin{bmatrix}
      a & b \\
      c & d
    \end{bmatrix} & = & \begin{bmatrix}
      a & b \\
      -c & -d
    \end{bmatrix}
\end{IEEEeqnarray*} so $b=-b$ and $c=-c$ and $b=c=0$.
\newpage
Using invertible matrix $\begin{bmatrix}
      0 & 1 \\
      1 & 0
\end{bmatrix}$ with  $\begin{bmatrix}
      a & 0 \\
      0 & d
\end{bmatrix}$
\begin{IEEEeqnarray*}{0rCl}
\begin{bmatrix}
      a & 0 \\
      0 & d
\end{bmatrix}
\begin{bmatrix}
      0 & 1 \\
      1 & 0
\end{bmatrix}
& = &
\begin{bmatrix}
      0 & a \\
      d & 0
\end{bmatrix}\\
\begin{bmatrix}
      0 & 1 \\
      1 & 0
\end{bmatrix}
\begin{bmatrix}
      a & 0 \\
      0 & d
\end{bmatrix}
& = &
\begin{bmatrix}
      0 & d \\
      a & 0
\end{bmatrix}
\end{IEEEeqnarray*}
and $a=d$.

Therefore the only matrices in $Z(G)$ are of the form $\begin{bmatrix}
      c & 0 \\
      0 & c
    \end{bmatrix}$ where $c\neq 0$.
\end{itemize}
   
    
\newpage
\subsection*{Problem 4: }
 For each of the following groups, decide whether the group is cyclic or not. Justify your answers.
    \begin{itemize}
        \item[a.] $\mathbb{Z}_{10}^*$
        \item[b.] $\mathbb{Z}_{12}^*$
        \item[c.] $\mathbb{Q}$ (with addition as operation)
        \item[d.] $\mathbb{R}^*$ (with multiplication as operation)
    \end{itemize}

    \subsection*{Answer 4:}
        \begin{itemize}
        \item[a.] $\mathbb{Z}_{10}^* = \{[1]_{10},[3]_{10},[7]_{10},[9]_{10}\}$: is cyclic\\
          $\langle[3]_{10}\rangle = \{[1]_{10},[3]_{10},[9]_{10},[27]_{10}\} =
           \{[1]_{10},[3]_{10},[9]_{10},[7]_{10}\} = \mathbb{Z}_{10}^*$
         \item[b.] $\mathbb{Z}_{12}^* = \{[1]_{12},[5]_{12},[7]_{12},[11]_{12}\}$: not cyclic, each
           $x \in \mathbb{Z}_{12}^*$
          squares to the identity.
        \item[c.] $\mathbb{Q}$: not cyclic: no integer multiple of $q\in \mathbb{Q}$ is in the interval $(0,|q|)$
        \item[d.] $\mathbb{R}^*$: not cyclic: a positive generator $r$ can't produce negative numbers,
          a negative generator can't produce $-r^2$.
    \end{itemize}

\newpage
\subsection*{Theoretical Problem 1: }
 Let $G$ be a group and $H \subset G$ a subset. Assume that for all $a, b \in H$, $a b^{-1}$ is also in $H$. 
    Prove that $H$ is a subgroup of $G$ (satisfies closure, identity and inverses).
\subsection*{Answer 1:}
\begin{itemize}
\item[] Associativity: inherited
\item[] Identity: $a\in H \Rightarrow aa^{-1} = e \in H$
\item[] Inverse: $e\in H \land a \in H \Rightarrow ea^{-1} = a^{-1} \in H$
\item[] Closure: $a,b \in H \Rightarrow a, b^{-1} \in H \Rightarrow a(b^{-1})^{-1} \in H \Rightarrow ab\in H$
\end{itemize}
\newpage
\subsection*{Theoretical Problem 2: }
Prove that every subgroup of $\mathbb{Z}$ is cyclic.

\subsection*{Answer 2:}
The two trivial subgroups of $\mathbb{Z}$ are cyclic.

Let $G$ be another subgroup of $\mathbb{Z}$
and let $n$ be the smallest positive integer in $G$. We assert that $G = n\mathbb{Z}$. If not,
then there exists $k\in G$ with $k>n$ such that $n$ does not divide $k$. But that means that
$d=\gcd(n,k) \in G$ as $\gcd(n,k)$ is a linear combination of $n$ and $k$. Either $d<n$ which
contradicts definition of $n$ or $d = n$ and $n\mid k$, another contracdiction.
\newpage
\subsection*{Theoretical Problem 3: }
Let $G$ be a group with $|G| = n$. Prove that $G$ is cyclic if and only if there exists
$x \in G$ with $o(x) = n$.
\subsection*{Answer 3:}
Lemma: $o(a) = k \iff |\langle a \rangle| = k$\\
Proof:
\begin{itemize}
\item[$\Rightarrow$] $|\langle a \rangle| > k$ is impossible since the sequence of $a^i$ repeats at $a^k=e$.\\
  $|\langle a \rangle| < k$ means $a^i = a^j$ for $i<j<k$ and $a^{j-i} = e$ with $j-i\neq k$.
  
\item[$\Leftarrow$]$|\langle a \rangle| = k$ means $\langle a \rangle$ has $k$ distinct elements and so
  $a^k =a^i$ for some $i<k$. But then $a^{k-i} = e$ and so $i=0$ and $o(a)=k$.
\end{itemize}
$G$ is cyclic means $G=\langle x \rangle$ for some $x\in G$ and $|\langle x \rangle| = n$ therefore
$o(x) = n$.

If $o(x) = n$ then $|\langle x \rangle| = n$ and any element of $G$ must be in $\langle x \rangle$ and
vice-versa and $\langle x \rangle = G$ and $G$ is cyclic.
\end{document}
% \begin{IEEEeqnarray*}{0rCl} \end{IEEEeqnarray*}
% \begin{IEEEeqnarray}{0rCl} \end{IEEEeqnarray}
% \begin{pmatrix} \end{pmatrix}
% \begin{vmatrix} \end{vmatrix}
% \text{tr\,}
% \text{det\,}
% \text{diag\,}
% \begin{itemize} \end{itemize}
% \begin{enumerate} \end{enumerate}
% \IEEEnonumber
% \setcounter{equation}{}
% \vert  \rangle    % ket
% \langle \vert     % bra
% \langle \vert  \vert  \rangle
