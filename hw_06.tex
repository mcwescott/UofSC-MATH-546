\documentclass[%
  10pt,
  letterpaper
]{article}
\usepackage{amsmath}
\usepackage{amssymb}
\usepackage{amsfonts}
\usepackage{IEEEtrantools}
\usepackage{MnSymbol}
\DeclareMathOperator\lcm{lcm}
% \usepackage{tikz}
\IEEEeqnarraydefcolsep{0}{\leftmargini}
\usepackage{bbm}
\setlength\parindent{0pt}
\setlength\parskip{4pt}
\begin{document}
\begin{center}
{UNIVERSITY OF SOUTH CAROLINA}
\end{center}
\large{MATH-546 Algebraic Structures I\hfill Homework 6}
\section*{Problems}
\subsection*{Problem 1: }
    \begin{enumerate}
        \item[(a)] List all the possible decomposition types of elements of \( A_7 \).
        \item[(b)] Find all the possible orders of elements of \( A_7 \). For each
          possible order, give an example of an element that has that order.
    \end{enumerate}

\subsection*{Answer 1:}

\begin{enumerate}
\item[(a)] For \(S_7\) the possible types are (7), (6), (5,2), (5), (4,3), (4,2), (4),
  (3,3), (3,2,2), (3,2), (3), (2,2,2), (2,2), (2), and ().  Of these (7), (5), (4,2), (3,3),
  (3,2,2), (3), (2,2), and () are even and possible in \(A_7\).
\item[(b)] These have orders of 7, 5, 4, 3, 6, 3, 2, and 0 respectively. Examples are
  (1 2 3 4 5 6 7), (1 2 3 4 5), (1 2 3 4)(5 6), (1 2 3)(4 5 6), (1 2 3)(4 5)(6 7), (1 2 3),
  (1 2)(3 4), and () respectively.
\end{enumerate}



\newpage
\subsection*{Problem 2: }
We say that a subgroup \( H \) of a group \( G \) is a normal subgroup if
the following condition is satisfied:
    \[
        \forall x \in G, \, \forall h \in H, \quad xhx^{-1} \in H
    \]
    \begin{enumerate}
        \item[(a)] Prove that \( A_n \) is a normal subgroup of \( S_n \).
        \item[(b)] Prove that \( H = \{ e, (1\,2) \} \) is not a normal subgroup of \( S_3 \).
    \end{enumerate}

\subsection*{Answer 2:}

\begin{enumerate}
\item[(a)] If \(\alpha \in A_n\) then \(\alpha\) has even parity. Any \(\sigma\in S_n\)
  has the same parity as its inverse and \(\sigma\alpha\sigma^{-1}\) has even parity
  being the product of 3 even permutations, or an even and two odd.
\item[(b)] Consider \((2\ 3)\in S_3\). Its inverse is itself and \((2\ 3)(1\,2)(2\ 3)=(1\ 3)\)
  which is not in \(H\).
    \end{enumerate}



\newpage
\subsection*{Problem 3: }
\begin{enumerate}
\item[(a)] Let \( G = A_4 \) and let \( H \) be the set consisting of the i
  dentity and all elements of \( G \) of order 2. Is \( H \) a subgroup of \( G \)?
  Prove your answer.
\item[(b)] Let \( G = S_4 \) and let \( H \) be the set consisting of the identity
  and all elements of \( G \) of order 2. Is \( H \) a subgroup of \( G \)?
  Prove your answer.
\end{enumerate}


\subsection*{Answer 3:}
\begin{enumerate}
\item[(a)] Only permutations of type (2,2) in \(G\) are both even and of order 2.
  They are self-inverses. Consider distinct \(\alpha, \beta \in G\).
  Then \(\alpha,\ \beta\)
  must have the forms \(\alpha=(a\ b)(c\ d)\) and \(\beta=(a\ c)(b\ d)\) for
  some distinct \(a, b, c, d\in \{1,2,3,4\}\). The product
  is \[(a\ b)(c\ d)(a\ c)(b\ d)=(a\ d)(b\ c)\] which is of order 2 and so \(H\) is closed.
\item[(b)] No: Counterexample: both (1 3) and (1 2)(3 4) have order 2 but their product
  (1 3)(1 2)(3 4) = (1 2 3 4) has
  order 4. so \(H\) is not closed.
\end{enumerate}



\newpage
\subsection*{Problem 4: }
 Prove that
    \[
        A_4 = \{ \sigma^2 \mid \sigma \in S_4 \}.
    \]
    \textit{Hint:} One inclusion is easy. For the other inclusion, consider
    the elements of \( A_4 \) written as either \( (i\, j\, k) \) or
    \( (i\, j)(k\, l) \) where \( i, j, k, l \in \{1,2,3,4\} \). You
    do not need to write each of the 12 elements of \( A_4 \), since
    they can all be represented in one of these forms. You need to
    prove that each one of these can be obtained as the square of some
    element of \( S_4 \).
    
\subsection*{Answer 4:}
\begin{itemize}
\item[\(\supseteq\):] \(\sigma^2\) is even.
\item[\(\subseteq\):] Consider distinct \(i,j,k,\ell \in \{1,2,3,4\}\). Then
  \[ (i\ j)(k\ \ell)=(i\ j\ k\ \ell)(i\ j\ k\ \ell)
  \] and \[(i\ j\ k) = (i\ k\ j)(i\ k\ j)
  \] and of course \[e = ee\]
  And that covers all possible elements of \(A_4\).
\end{itemize}



\newpage
\subsection*{Problem 5: }
Let \( H \) be a subgroup of \( A_4 \) and let \( a = (1\,2)(3\,4),
\; b = (1\,2\,3) \in A_4 \). If \( a, b \in H \), prove that \( H \)
must be equal to the entire \( A_4 \). 

\textit{Hint:} Use the closure property to find a few more of the
elements of \( H \); if you can find at least 7 different elements
of \( H \), you can then invoke Lagrange’s theorem.
\subsection*{Answer 5:} It's given that \(e, a, b \in H\). Note that \(a^2=e\) and \(b^3=e\).
By closure
the following are in \(H\):
\begin{IEEEeqnarray*}{0rCl}
  e & = & (\,)\\
  a & = & (1\ 2)(3\ 4)\\
  b & = & (1\ 2\ 3)\\
  b^2 & = & (1\ 3\ 2) \\
  ab  & = & (2\ 4\ 3)\\
  ab^2 & = & (1\ 4\ 3)\\
  ba & = & (1\ 3\ 4)\\
  b^2a & = & (2\ 3\ 4)\\
  aba & = & (1\ 4\ 2)\\
  ab^2a & = & (1\ 2\ 4)\\
  bab^2 & = & (1\ 4)(2\ 3)\\
  abab^2 & = & (1\ 3)(2\ 4)
\end{IEEEeqnarray*}

We could have stopped at 7 and argued that the order of \(H\) is 7 or greater and
must divide 12, the order of \(A_4\) and so the order of \(H\) is 12 and \(H=A_4\)
but we only needed 5 more to finish the list.

\newpage
\section*{ Theoretical Questions}

\subsection*{Problem 6: }
Prove that \( |A_n| = n!/2 \).
\subsection*{Answer 6:}
Let \(O_n = S_n\setminus A_n\). Clearly \(O_n\) is the set of all odd
permutations with \(A_n\cup O_n = S_n\) and \(A_n\cap O_n = \emptyset\)
and so \(|A_n|+|O_n|=|S_n|=n!\). So all we need show is that \(|A_n|=|O_n|\).

We do so by constructing a bijection \(F:A_n\mapsto O_n\) with inverse
\(F^{-1}:O_n\mapsto A_n\). \(\forall\alpha\in A_n, \forall\beta\in O_n, F(\alpha) =
\alpha\,(1\ 2)\in O_n\) and \(F^{-1}(\beta) = \beta\,(1\ 2) \in A_n\). \[F^{-1}(F(\alpha)) =
\alpha\,(1\ 2)\,(1\ 2) = \alpha\] and \[F(F^{-1}(\beta) = \beta\,(1\ 2)\,(1\ 2) = \beta\]
and \(F\) is indeed a bijection. Therefore \( |A_n| = n!/2 \). \(\square\)

\newpage
\subsection*{Problem 7: }
 Prove that \( A_n \) is a subgroup of \( S_n \).
\subsection*{Answer 7:}
\(\forall a,b\in A_n\): \(a\) and \(b\) are both even and so therefore is \(b^{-1}\).
And then \(ab^{-1}\) is even and \(ab^{-1}\in A_n\). \(\square\)


\end{document}
% \begin{IEEEeqnarray*}{0rCl} \end{IEEEeqnarray*}
% \begin{IEEEeqnarray}{0rCl} \end{IEEEeqnarray}
% \begin{pmatrix} \end{pmatrix}
% \begin{vmatrix} \end{vmatrix}
% \text{tr\,}
% \text{det\,}
% \text{diag\,}
% \begin{itemize} \end{itemize}
% \begin{enumerate} \end{enumerate}
% \IEEEnonumber
% \setcounter{equation}{}
% \vert  \rangle    % ket
% \langle \vert     % bra
% \langle \vert  \vert  \rangle
